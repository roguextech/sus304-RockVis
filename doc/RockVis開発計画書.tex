\documentclass[10pt,a4paper]{jsarticle}

%- Figure/Table Setting
\usepackage[dvipdfmx]{graphicx,xcolor}
\graphicspath{./Figs/}
\usepackage{float}

%- Math Setting
\usepackage{amsmath,amssymb}

%- Font Setting
\usepackage[T1]{fontenc}
\usepackage{textcomp}
\usepackage[sc]{mathpazo}
\usepackage[scaled]{helvet}
\renewcommand{\ttdefault}{lmtt}
\usepackage{otf}

\renewcommand{\figurename}{Fig.\,\,}
\renewcommand{\tablename}{Table\,\,\,}

\begin{document} % 文書本体の開始
	
	\title{開発計画書}
	\author{田中 進夢}
	\maketitle
	
	
	%- Main Contents
	\section{開発物}
	飛翔ログ解析ツール
	
	\section{開発責任者}
	情報処理班 田中進夢
	
	\section{依頼者}
	計測制御班 小川誠仁
	
	\section{開発内容}
	打上実験にて計器より得られた各種データから飛翔履歴を解析、可視化ツールの開発
	
	\section{開発期間}
	未指定
	
	\section{設計方針}
	Python を使用したツールを作成。計器から得られた digit データから各種物理量の変換を行う。気圧から
	気圧高度、加速度および角速度から INS 飛翔履歴を算出する。変換したデータは csv で書き出し、可視化には matplotlib を使用する。
	
	
	
	
	\newpage
\end{document} % 文書終了